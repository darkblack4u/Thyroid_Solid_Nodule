\documentclass[journal]{IEEEtran}
\usepackage{graphicx}
\usepackage{cite}


% correct bad hyphenation here
\hyphenation{op-tical net-works semi-conduc-tor}


\begin{document}
\title{Bare Demo of IEEEtran.cls\\ for IEEE Journals}

\author{Michael~Shell,~\IEEEmembership{Member,~IEEE,}
        John~Doe,~\IEEEmembership{Fellow,~OSA,}
        and~Jane~Doe,~\IEEEmembership{Life~Fellow,~IEEE}% <-this % stops a space
\thanks{M. Shell was with the Department
of Electrical and Computer Engineering, Georgia Institute of Technology, Atlanta,
GA, 30332 USA e-mail: (see http://www.michaelshell.org/contact.html).}% <-this % stops a space
\thanks{J. Doe and J. Doe are with Anonymous University.}% <-this % stops a space
\thanks{Manuscript received April 19, 2005; revised August 26, 2015.}}


% The paper headers
\markboth{Journal of \LaTeX\ Class Files,~Vol.~14, No.~8, August~2015}%
{Shell \MakeLowercase{\textit{et al.}}: Bare Demo of IEEEtran.cls for IEEE Journals}

% make the title area
\maketitle

% As a general rule, do not put math, special symbols or citations
% in the abstract or keywords.
\begin{abstract}
The abstract goes here.
\end{abstract}

% Note that keywords are not normally used for peerreview papers.
\begin{IEEEkeywords}
IEEE, IEEEtran, journal, \LaTeX, paper, template.
\end{IEEEkeywords}

% \IEEEpeerreviewmaketitle

\section{Introduction}
\IEEEPARstart{T}{his} demo file is intended to serve as a ``starter file''
for IEEE journal papers produced under \LaTeX\ using
IEEEtran.cls version 1.8b and later.
 
I wish you the best of success.

% needed in second column of first page if using \IEEEpubid
%\IEEEpubidadjcol
\section{Related work}\label{section_related_work}

\subsubsection{Medical Image Segmentation}

\subsubsection{Medical Image Segmentation}






Subsubsection text here.

\section{Conclusion}
The conclusion goes here.





% if have a single appendix:
%\appendix[Proof of the Zonklar Equations]
% or
%\appendix  % for no appendix heading
% do not use \section anymore after \appendix, only \section*
% is possibly needed

% use appendices with more than one appendix
% then use \section to start each appendix
% you must declare a \section before using any
% \subsection or using \label (\appendices by itself
% starts a section numbered zero.)
%


\appendices
\section{Proof of the First Zonklar Equation}
Appendix one text goes here.

% you can choose not to have a title for an appendix
% if you want by leaving the argument blank
\section{}
Appendix two text goes here.


% use section* for acknowledgment
\section*{Acknowledgment}


The authors would like to thank...

\cite{IEEEexample:biblatex}

\bibliographystyle{IEEEtran}
\bibliography{IEEEabrv,IEEEexample}

\begin{IEEEbiography}{Michael Shell}
Biography text here.
\end{IEEEbiography}

\end{document}


